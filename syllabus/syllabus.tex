% (The MIT License)
%
% Copyright (c) 2023 Aliaksei Bialiauski
%
% Permission is hereby granted, free of charge, to any person obtaining a copy
% of this software and associated documentation files (the 'Software'), to deal
% in the Software without restriction, including without limitation the rights
% to use, copy, modify, merge, publish, distribute, sublicense, and/or sell
% copies of the Software, and to permit persons to whom the Software is
% furnished to do so, subject to the following conditions:
%
% The above copyright notice and this permission notice shall be included in all
% copies or substantial portions of the Software.
%
% THE SOFTWARE IS PROVIDED 'AS IS', WITHOUT WARRANTY OF ANY KIND, EXPRESS OR
% IMPLIED, INCLUDING BUT NOT LIMITED TO THE WARRANTIES OF MERCHANTABILITY,
% FITNESS FOR A PARTICULAR PURPOSE AND NONINFRINGEMENT. IN NO EVENT SHALL THE
% AUTHORS OR COPYRIGHT HOLDERS BE LIABLE FOR ANY CLAIM, DAMAGES OR OTHER
% LIABILITY, WHETHER IN AN ACTION OF CONTRACT, TORT OR OTHERWISE, ARISING FROM,
% OUT OF OR IN CONNECTION WITH THE SOFTWARE OR THE USE OR OTHER DEALINGS IN THE
% SOFTWARE.

\documentclass[nobrand,anonymous,nodate,nosecurity]{../solvd}
\usepackage{multicol}
\usepackage{href-ul}
\usepackage{ffcode}
\begin{document}

{
    \sffamily{\bfseries\Large Microserivces and beyoynd}\\
    Lectures about microservices by \href{h1alexbel.github.io/about-me.html}{Aliaksei Bialiauski}

    \begin{abstract}
        This is a series of lectures related to microservices development.
        The lectures provide basics and includes practical best practices for each topic.
    \end{abstract}

    \textbf{Prerequisites} (it is expected that listener knows this):
    \begin{itemize}
        \item Spring Framework
        \item PostgreSQL
        \item Containerization and Docker
    \end{itemize}

    \section*{Course Structure}
    \begin{itemize}
        \item API size, database-per service
        \item Service discovery
        \item Gateways
        \item Load balancing
        \item Communication: sync vs async
        \item Messaging
        \item Data formats: XML, JSON, protobuf, etc.
        \item Saga, Event sourcing, CQRS
        \item NoSQL
        \item Caching
        \item Rate-limiting
        \item Load shedding
        \item Circuit breaker
        \item Infrastructure, autoscaling
    \end{itemize}
    Toolset we are going to use:
    \begin{itemize}
        \item \href{k8s.io}{K8s}
        \item \href{kafka.org}{Kafka}
        \item \href{redis.com}{Redis}
        \item \href{mongo.com}{MongoDB}
        \item \href{reactor.org}{Spring Reactive}
        \item \href{terraform.io}{Terraform}
    \end{itemize}
    \newpage
    \textbf{Application we are going to build}:
    \href{https://peopleforce.io}{Peopleforce} analogue for Resource managers and Project managers in the \href{https://solvd.com}{Solvd} company
    \\
    and use:
    \begin{itemize}
        \item Saga
        \item Message broker
        \item NIO
        \item External cache
        \item L7 and L4 load balancing
        \item Service mesh
        \item DevOps practices
    \end{itemize}
\end{document}